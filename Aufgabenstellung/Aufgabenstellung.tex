%--------------------------------------------------------------------------------
\documentclass{article}
\usepackage[latin1]{inputenc}
%--------------------------------------------------------------------------------

\begin{document}

\section{Einzuf�hrende Strukturen}
\subsection{Klasse f�r Teilgebiete}
F�hre die Klasse
\begin{itemize}
\item \texttt{class SubDomain\{...\}}
\end{itemize}
ein. 
In diser soll die Prozessnummer, die Anzahl der gestarteten Prozesse,
die Position des Prozesses im Prozessgitter (\texttt{ip[DIM])}), die
Anzahl der Teilgebiete, die Prozessnummern der Nachbarprozesse
(\texttt{ip\_lower[DIM], ip\_upper[DIM]}), die Breite der Randbord�re
(\texttt{ic\_start[DIM]}), der erste lokale Index in der oberen
Randbord�re (\texttt{ic\_stop[DIM]}), die Anzahl der Zellen im gesamten
Teilgebiet einschlie�lich der Randbord�ren, die Kantenl�nge der Zellen
und der globale Index der ersten Zelle des Teilgebiets
(\texttt{ic\_lower\_global[DIM]}) gespeichert sein. 

Dazu muss folgendes Bestimmt werden: 

\begin{itemize}
\item Die globalen Zellindizes des Teilgebietes m�ssen bestimmt
  werden. 
\item Die Position im Prozessgitter muss berechnet werden.
\item In Abh�ngigkeit von der Randbedingungen m�ssen die
  Prozessnummern der Nachbarprozesse bestimmt werden. 
\item Die Zahl der Zellen muss bestimmt werden. 
\item Die Prozesse m�ssen auf die Teilgebiete verteilt werden und die
  Position im Prozessgitter muss mit der Prozessnummer
  (\texttt{myrank}) in Zusammenhang gebracht werden. 
\end{itemize}





\end{document}
